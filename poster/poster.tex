\documentclass[a0paper,portrait]{baposter}

\usepackage{relsize}
\usepackage{url}
\usepackage{multicol}
\usepackage{graphicx}
\usepackage{fontawesome5}
\usepackage{amsmath,amssymb}

\usepackage{tikz}
\usetikzlibrary{positioning, shapes.geometric}

\graphicspath{{assets}}

\begin{document}

%%% General Poster Settings %%%%%%%%%%%%%%%%%%%%%%%%%%%%%%%%%%%%%%%%%%%%%%%%%%%
%%%%%% Eye Catcher, Title, Authors and University Images %%%%%%%%%%%%%%%%%%%%%%
\begin{poster}{
  columns=2,
	grid=false,
	borderColor=uniblue,
	headerColorOne=uniblue,
	headerColorTwo=uniblue,
	headerFontColor=white,
  headerheight=14em,
	boxColorOne=white,
  boxpadding=1em,
	headershape=rounded,
	headerfont=\Large\textsf,
	textborder=rounded,
	background=shadetb,
  bgColorOne=uniblue!10,
  bgColorTwo=uniblue!30,
	headerborder=open,
  boxshade=plain,
  eyecatcher=true
}
%%% Eye Cacther %%%%%%%%%%%%%%%%%%%%%%%%%%%%%%%%%%%%%%%%%%%%%%%%%%%%%%%%%%%%%%%
{
}
%%% Title %%%%%%%%%%%%%%%%%%%%%%%%%%%%%%%%%%%%%%%%%%%%%%%%%%%%%%%%%%%%%%%%%%%%%
{\smaller Pedagogy behind crypto CTF challenges \\ and a way to improve them}
%%% Authors %%%%%%%%%%%%%%%%%%%%%%%%%%%%%%%%%%%%%%%%%%%%%%%%%%%%%%%%%%%%%%%%%%%
{
  \vspace{1em}
  Emanuel Nicolás Herrador\\
	{\smaller emanuel.nicolas.herrador@unc.edu.ar}
}
%%% Logo %%%%%%%%%%%%%%%%%%%%%%%%%%%%%%%%%%%%%%%%%%%%%%%%%%%%%%%%%%%%%%%%%%%%%%
{\begin{minipage}{28.0em}
    \vspace{-4em}
    \includegraphics[height=4.4em]{famaf-unc-logo.png}
  \end{minipage}
}

%%% What's CTF? %%%%%%%%%%%%%%%%%%%%%%%%%%%%%%%%%%%%%%%%%%%%%%%%%%%%%%%%%%%%%%%%%%
\headerbox{What's CTF?}{name=whats-ctf,column=0,row=0}{
  \begin{itemize}
    \item Capture The Flag is a competition for cybersecurity field.
    \item Problems are solved obtaining \textit{flags} (words with specific format)
      and they are given when we broke a weak protocol, corrupted file or system for it.
    \item Problem categories:
  \end{itemize}
  \begin{center}
    \begin{tikzpicture}[
      node distance=0.4cm and 0.5cm,
      every node/.style={
        draw=uniblue!70,
        shape=rectangle,
        fill=uniblue!70,
        text=white,
        minimum width=2.2cm,
        minimum height=1cm,
        font=\footnotesize\bfseries,
        align=center
      }
    ]

      % 1st column
      \node (a1) {Web \\ Exploitation};
      \node[below=of a1] (a2) {Binary \\ Exploitation};
      \node[below=of a2] (a3) {Reverse \\ Engineering};

      % 2nd column
      \node[right=of a1] (b1) {Cryptography};
      \node[below=of b1] (b2) {Forensics};
      \node[below=of b2] (b3) {Hardware};

      % 3rd column
      \node[right=of b1] (c1) {OSINT};
      \node[below=of c1] (c2) {Misc};
      \node[below=of c2] (c3) {PPC};

    \end{tikzpicture}
  \end{center}
}

%%% Importance %%%%%%%%%%%%%%%%%%%%%%%%%%%%%%%%%%%%%%%%%%%%%%%%%%%%%%%%%%%%%%%%%%
\headerbox{Importance}{name=importance,column=0,below=whats-ctf}{
  \begin{itemize}
    \item Practical experience and exposure to real-world vulnerabilities or attacks.
    \item Teamwork and community across social websites, jobs and universities.
      A way to combine all the worlds of learning (self learning, academy, industry).
    \item Worldwide competitions: \textit{DEF CON}, \textit{ICC}, \textit{ECSC},
      \textit{Pwn20wn}, \textit{Google CTF}.
    \item Gamification way to learn and teach:
      \begin{itemize}
        \item To improve playing in cybersecurity.
        \item Discovering new attacks or ways to think problems.
        \item Mixing theory, practice and tricks.
      \end{itemize}
    \item Used in a lot of universities for competitions, teach and generate more 
      interest in cybersecurity field.
  \end{itemize}
}

%%% Complexity for a problemsetter %%%%%%%%%%%%%%%%%%%%%%%%%%%%%%%%%%%%%%%%%%%%%%%%%%%%%%%%%%%%%%%%%%
\headerbox{Complexity for a problemsetter}{name=complexity-for-a-problemsetter,column=0,below=importance}{
  \begin{itemize}
    \item Find interesting topics for problems.
    \item Being creative and make it useful to learn.
    \item Make it neither too easy nor too difficult.
    \item Feasible to solve in the way we thought and, also, ensure it as unique to test
      specific objective skills for evaluation.
    \item Avoid being solved only with pre-existent automated tools or LLMs (as with \textit{ico}
      problem of DEFCON finals 2025).
  \end{itemize}
}

%%% How would you create a CTF challenge in your study field? %%%%%%%%%%%%%%%%%%%%%%%%%%%%%%%%%%%%%%%%%%%%%%%%%%%%%%%%%%%%%%%%%%
\headerbox{How would you create a chall in your study field?}{name=ctfs-suggestions,column=0,below=complexity-for-a-problemsetter}{
  \begin{itemize}
    \item CTF challs are a useful way to learn and teach each area of study.
    \item Do you want to try creating one for teaching or share with partners?
    \item Tell me more! \underline{I'll be doing feedback until 31/10} (form closes on 10).
  \end{itemize}
  \begin{center}
    \includegraphics[width=0.382\columnwidth]{qr-google-forms.png}
  \end{center}
}

%%% Samsara %%%%%%%%%%%%%%%%%%%%%%%%%%%%%%%%%%%%%%%%%%%%%%%%%%%%%%%%%%%%%%%%%%%%%
\headerbox{Crypto CTF challenge example: Samsara}{name=samsara,column=1,row=0}{
  \begin{center}
    {\large\textbf{International Cybersecurity Challenge 2024}} \\
    (Santiago, Chile)

    \vspace{0.5em}
    {\small\textit{Topics}: Linear Congruential Generator, Non-interactive Girault's proof 
    of knowledge (with Fiat-Shamir transformation) and Frozen Heart Vulnerability.}
  \end{center}
  \vspace{-2em}
  \hhrule

  \vspace{-0.5em}
  {\large\textbf{Statement}}

  Break the following protocol to capture the flag:
  \begin{center}
    \includegraphics[width=0.95\columnwidth]{samsara-protocol-black.png}
  \end{center}
  
  \vspace{-2em}
  \hhrule

  \vspace{-0.5em}
  {\large\textbf{Solution idea}}: We've two main parts. Let see each one separately.

  \vspace{0.5em}
  \textbf{First part} (Linear Congruential Generator)
  \begin{itemize}
    \item LCG with unknown parameters and are given non consecutive terms of
      this sequence (with its positions).
    \item Let be $s_k \! = \! f^{(k)}(s)$ the terms we have.
      An useful property to use is $s_A - s_B \equiv (a^A-a^B)\left(s+\frac{b}{a-1}\right)\pmod{p}$.
    \item Let $T \! = \! pX, S \! = \! pY$, then $\gcd(S,T) \! = \! p \! \iff \! \gcd(X,Y) \! = \! 1$.
      Since the probability that two random numbers are coprime is $61\%$,
      we've a good chance to get $p$ with a little quantity of $(T,S)$ tries.
    \item Aiming $R : R \! \neq \! 0 \land R \! \equiv \! 0 \pmod{p}$,
      we'll search for positions values $A,B,C,D,E,F,G,H \in \left\{\sum_{i=0}^k M_i\right\}_{k=0}^{\# M}$ such that:
      \begin{equation*}
          0 \equiv (s_A-s_B)(s_C-s_D)-(s_E-s_F)(s_G-s_H)
            \equiv B^2 K\pmod{p}
      \end{equation*}
      where $B = s+\frac{b}{a-1}, K = (a^A-a^B)(a^C-a^D)-(a^E-a^F)(a^G-a^H)$.
    \item Since $B \equiv 0 \pmod{p}$ is unlikely to happen, we aim for $p|K$. For that:
      \begin{equation*}
        \{A+C,B+D\} = \{E+G,F+H\} \land \{A+D,B+C\} = \{E+H,F+G\}
      \end{equation*}

      \vspace{-0.8em}
    \item Let $m = \#M + 1$. The search can be done with bruteforce in $O(m^8)$,
      but searching intelligently in $O(m^6)$.
      The total number of useful combinations is $288$ and we're able 
      to get $p$ and also $q = \frac{n}{p}$.
  \end{itemize}

  \textbf{Second part} (Girault's identification non-interactive protocol)
  
  \begin{itemize}
    \item $h$ isn't used in hash and it's a weakness (\textit{Frozen Heart Vulnerability}).
    \item Choosing random values $u,z$, we've $e = \text{Hash}(g,N,u)$ and we aim for
      $h \equiv (u \cdot g^{-z})^{e_\text{inv}} \pmod{n}$ where $e_\text{inv} \equiv e^{-1} \pmod{\phi(n)}$
      because:

      \vspace{-1em}
      \begin{equation*}
        g^z \cdot h^e \equiv g^z \cdot \left((u \cdot g^{-z})^{e_\text{inv}}\right)^e \equiv g^z \cdot u \cdot g^{-z} \equiv u \pmod{n}
      \end{equation*}


      \vspace{-0.6em}
    \item We can compute $e_\text{inv}$ easily with \textit{Extended Euclidean algorithm} because we know
      $\phi(n)$.
      If it doesn't exist, try another $u,z$ random values.
    \item Sending these values to verifier, it should give us the flag message.
  \end{itemize}

  \textbf{Complete solution and implementation}

  \begin{center}
    \raisebox{-0.3em}{\Large\faGithub}\quad \url{github.com/helcsnewsxd/latincrypt-2025}
  \end{center}
}

\end{poster}
\end{document}
