\documentclass{article}
\usepackage[utf8]{inputenc}
\usepackage[english]{babel}
\usepackage{geometry}
  \geometry{
    a4paper,
    total = {170mm, 257mm},
    left = 20mm,
    top = 20mm,
  }
\usepackage{graphicx}
\usepackage{titling}
\usepackage{hyperref}

\setlength{\headheight}{14pt}

\usepackage{amsmath}
\usepackage{amsthm}
\usepackage{amssymb}

% Draw
\usepackage{tikz}
\usetikzlibrary{positioning, arrows.meta, shapes.geometric, calc}
\tikzset{
  >={Stealth[length=3mm]},
  box/.style={draw, rounded corners, fill=blue!10, inner sep=6pt, text width=5cm, align=center, minimum height=3cm},
  data/.style={draw, ellipse, fill=green!10, inner sep=4pt, align=center},
}

% Code environment 
\usepackage{minted}
\usepackage{caption}

\setminted{
  mathescape,
  linenos,
  numbersep=3pt,
  frame=lines,
  framesep=1mm,
  baselinestretch=1,
  fontsize=\footnotesize,
  breaklines=true,
  xleftmargin=1em,
  xrightmargin=0em,
}

\newcounter{code}
\newcommand{\listofcode}{\listof{code}{List of Codes}}
\captionsetup[code]{labelfont=bf, name=Code}

\newcommand{\codefile}[4][python]{%
  \refstepcounter{code}%
  \captionsetup[listing]{name=Code}%
  \captionof{listing}{#2}%
  \label{#3}%
  \inputminted{#1}{#4}%
  }

% Pseudocode environment
\usepackage[ruled,vlined]{algorithm2e}

% Math environments
\usepackage{aliascnt}
\renewcommand{\theequation}{\thesection.\arabic{equation}}

\theoremstyle{remark}
\newtheorem*{remark}{Remark}

\newenvironment{iproof}[1][Proof idea]
  {
    \begin{proof}[#1]
      \renewcommand{\qedsymbol}{}
  }
  {
  \end{proof}
  }

% Commands 
\usepackage{xparse}

\usepackage{xcolor}
\newcommand{\todo}[1]{
  \begin{quotation}
    \textcolor{red}{\textbf{TODO: }#1}
  \end{quotation}
}

\newcommand{\N}{\mathbb{N}}
\NewDocumentCommand{\R}{g g}{
  \IfNoValueTF{#1}
    {\mathbb{R}}
    {
    \IfNoValueTF{#2}
      {\mathbb{R}^{#1}}
      {\mathbb{R}^{#1 \times #2}}
    }
}

\newcommand{\Z}{\mathbb{Z}}
\newcommand{\F}{\mathbb{F}}

\newcommand{\Pm}{\mathcal{P}}
\newcommand{\Vm}{\mathcal{V}}
\newcommand{\Lm}{\mathcal{L}}
\newcommand{\Em}{\mathcal{E}}
\newcommand{\Km}{\mathcal{K}}

\newcommand{\abs}[1]{\left| #1 \right|}
\newcommand{\gen}[1]{\left\langle #1 \right\rangle}

\newcommand{\getrandom}{\overset{\$}{\leftarrow}}

% Document
\usepackage{fancyhdr}
  \fancypagestyle{plain}{
    \fancyhf{}
    \fancyhead[L]{\thetitle}
    \fancyhead[R]{\theauthor}
  }

\makeatletter
\newcommand{\subtitle}[1]{\gdef\@subtitle{#1}}
\newcommand{\@subtitle}{}
\makeatother

\makeatletter
\def\@maketitle{
  \newpage 
  \null 
  \vskip 1em 
  \begin{center}
    \let \footnote \thanks 
      {\LARGE \@title \par}
      \vskip 0.5em
      {\large \@subtitle \par}
      \vskip 1em
      {\large \@date}
  \end{center}
  \par 
  \vskip 1em 
}
\makeatother


\title{ASCrypto 2025: Introduction to Proving Systems}
\author{Emanuel Nicolás Herrador}
\newcommand{\speaker}{Arantxa Zapico}
\date{September 29 of 2025}

\begin{document}
  \maketitle 
  \noindent\begin{tabular}{@{}ll}
    Author & \theauthor \\
    Speaker & \speaker
  \end{tabular}

  \section{Interactive proofs}
  The idea is to prove something to another entity efficiently without proving it 
  at all (complete).
  It is made interactive.
  It must satisfy:
  \begin{itemize}
    \item Completeness: If something is indeed true and both (prover and verifier), follow 
      the procedure, verifier accept 
    \item Soundness: If something is false, then verifier rejects with overwhelming probability 
    \item Zero-Knowledge: The verifier does not learn anything but the truth of something
  \end{itemize}

  More formally, we define something as follows.
  Let be $R = \{(x,y) : \dots\}$ a PT relationship, and ``something is true'' if 
  $x \in \mathcal{L}_R$ where $L$ is the language of $R$.

  Prover sends messages to verifier and verifier sends challenges to our prover to prove that 
  it's true something.
  The challenge are random (as possible) to avoid deshonest provers.

  There're some points to take in account like efficiency.
  Things like public parameter size (pp), proof size ($|m_1| + \dots + |m_k|$), prover time
  and verifier time.

  Let's see some properties that interactive proofs can have.
  For that, we consider $\Pm$ as honest prover and $\Pm^*$ as malicious prover.

  \paragraph{Succinctness}
  This is the first point about a SNARK.
  We need a proving system to be succinct in communication and in verification.
  I.e., $\sum |m_i| << \abs{w}$ and $\text{time}(\mathcal{V}) << \text{time}_R(x,w)$
  respectively.

  \paragraph{(Perfect) Completeness}
  If $x \in \mathcal{L}_R$ and $\Pm$ follow the procedure, $\Vm$ accepts.
  The probability must be $1$ if we want perfect completeness.
  Formally, $Pr[\langle \Pm(pp,(x,w), \Vm(pp,x))] = 1$.

  \paragraph{(Computational) Soundness}
  If $x\notin\mathcal{L}_R$ then $\Vm$ rejects with overwhelming probability.
  I.e., if $\nexists w : (xm,w) \in R$ then $\Vm$ rejects it with overwhelming probability.
  Formally, $Pr[\langle \Pm^*(pp,x),\Vm(pp,x)\rangle \leq \text{negl}(\lambda)$

  \paragraph{Knowledge-soundness}
  There exists a PT algorithm $\Em$, the extractor, such that for every malicious prover $\Pm^*$,
  then $Pr[(x,w) \in R : w \leftarrow \Em^{\Pm^*}(x)] - Pr[\langle \Pm^*(x),\Vm(x)\rangle = 1] \leq \text{negl}(\lambda)$

  An argument that satisfy Knowledge-soundness is an argument of knowledge.
  I.e., Knowledge-soundness is more powerful that soundness.

  \subsection{SNARK(G)s}
  It's one way to construct it.
  
  \paragraph{Tool 1: Interactive Oracle}
  The prover sends message to an oracle and verifier ask questions to this oracle.
  Then, verifier doesn't learns anything about $m_i$ because its questions are asked by the oracle.
  The oracle works as an intermediate between them and we assume it as powerful and trusted.

  \paragraph{Tool 2: Functional Commitment Scheme}
  We can't trust to an oracle.
  Therefore, we've to use a commitment scheme to replace this oracle when $\Pm$ sends 
  commitments and also make computations of $f$; $\Pm$ also sends the proof.
  The scheme is as follows: $\Pm$ sends a commitment of $m_1$, then $\Vm$ ask for $f(m_1,\alpha_1)$
  and $\Pm$ reply with $y \leftarrow f(m_1,\alpha_1)$ and $\pi \leftarrow \text{Open}(f, y)$ as the proof.
  Finally, $\Vm$ can do the verification with $\text{Verify}(y,\text{com}_1,\alpha_1,\pi)$.

  Notice that the succinctness of this scheme is based in the commitments (as for example, 
  proof size is the size of commits and $pp \leftarrow CS \cdot \mathcal{K}$).

  As Oracle is an idealized model, here we don't have cryptographic assumptions.
  However, we've it in the commitment scheme.

  \paragraph{From interactive to non-interactive proofs}
  With that we construct an interactive succinct argument and we want to construct a non-interactive 
  succinct argument.
  For that we'll use a hash function $H : \{0,1\}^* \to \{0,1\}^{256}$.
  And we want that for this function the following problems are harder to solve (computationally):
  \begin{itemize}
    \item Collision resistant: Find $x,y : H(x) = H(y)$
    \item Pre-image resistant: Given $z$, find $x : H(x) = z$
    \item Second pre-image resistant : Given $x$, find $y : H(x) = H(y)$
  \end{itemize}

  Here, all the messages (commitments) and challenge answers are sent by $\Pm$ once.
  Therefore, we want a way for $\Pm$ to generate random challenges to solve and prove that 
  they are random.

  Here we'll use \textbf{Fiat-Shamir Heuristic}.
  The challenge is created with the hash of all the public information gotten before.
  I.e., for example, $a_i = H(x,\text{com}_1,\dots,\text{com}_i)$.
  Let's say that $\pi$ is all the proof sent by $\Pm$.
  Then, it's secure under the random oracle model.

  We've for knowledge soundness:
  \begin{equation*}
    \begin{aligned}
      &pp \leftarrow \mathcal{K} \\ 
      &(x,\pi) \leftarrow \Pm^*(pp) \\ 
      &w \leftarrow \Em(pp, x, \pi) \\
      &Pr\left[(x,w) \notin R \land \Vm(pp,x,\pi) = 1\right] \leq \text{negl}(\lambda)
    \end{aligned}
  \end{equation*}

  \subsection{Short summary}
  We saw:
  \begin{itemize}
    \item SNARK: Succinct Non-interactive Argument of Knowledge 
    \item SNARG: Succinct Non-interactive Argument
    \item Efficiency: prover/verifier time, proof/pp size 
    \item Security: Setup (trusted/transparent), model (ROM) and assumptions (discrete log)
  \end{itemize}
  And most of it depends on the commitment scheme.
\end{document}
