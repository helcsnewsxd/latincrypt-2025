\documentclass{article}
\usepackage[utf8]{inputenc}
\usepackage[english]{babel}
\usepackage{geometry}
  \geometry{
    a4paper,
    total = {170mm, 257mm},
    left = 20mm,
    top = 20mm,
  }
\usepackage{graphicx}
\usepackage{titling}
\usepackage{hyperref}

\setlength{\headheight}{14pt}

\usepackage{amsmath}
\usepackage{amsthm}
\usepackage{amssymb}

% Draw
\usepackage{tikz}
\usetikzlibrary{positioning, arrows.meta, shapes.geometric, calc}
\tikzset{
  >={Stealth[length=3mm]},
  box/.style={draw, rounded corners, fill=blue!10, inner sep=6pt, text width=5cm, align=center, minimum height=3cm},
  data/.style={draw, ellipse, fill=green!10, inner sep=4pt, align=center},
}

% Code environment 
\usepackage{minted}
\usepackage{caption}

\setminted{
  mathescape,
  linenos,
  numbersep=3pt,
  frame=lines,
  framesep=1mm,
  baselinestretch=1,
  fontsize=\footnotesize,
  breaklines=true,
  xleftmargin=1em,
  xrightmargin=0em,
}

\newcounter{code}
\newcommand{\listofcode}{\listof{code}{List of Codes}}
\captionsetup[code]{labelfont=bf, name=Code}

\newcommand{\codefile}[4][python]{%
  \refstepcounter{code}%
  \captionsetup[listing]{name=Code}%
  \captionof{listing}{#2}%
  \label{#3}%
  \inputminted{#1}{#4}%
  }

% Pseudocode environment
\usepackage[ruled,vlined]{algorithm2e}

% Math environments
\usepackage{aliascnt}
\renewcommand{\theequation}{\thesection.\arabic{equation}}

\theoremstyle{remark}
\newtheorem*{remark}{Remark}

\newenvironment{iproof}[1][Proof idea]
  {
    \begin{proof}[#1]
      \renewcommand{\qedsymbol}{}
  }
  {
  \end{proof}
  }

% Commands 
\usepackage{xparse}

\usepackage{xcolor}
\newcommand{\todo}[1]{
  \begin{quotation}
    \textcolor{red}{\textbf{TODO: }#1}
  \end{quotation}
}

\newcommand{\N}{\mathbb{N}}
\NewDocumentCommand{\R}{g g}{
  \IfNoValueTF{#1}
    {\mathbb{R}}
    {
    \IfNoValueTF{#2}
      {\mathbb{R}^{#1}}
      {\mathbb{R}^{#1 \times #2}}
    }
}

\newcommand{\Z}{\mathbb{Z}}
\newcommand{\F}{\mathbb{F}}

\newcommand{\Pm}{\mathcal{P}}
\newcommand{\Vm}{\mathcal{V}}
\newcommand{\Lm}{\mathcal{L}}
\newcommand{\Em}{\mathcal{E}}
\newcommand{\Km}{\mathcal{K}}

\newcommand{\abs}[1]{\left| #1 \right|}
\newcommand{\gen}[1]{\left\langle #1 \right\rangle}

\newcommand{\getrandom}{\overset{\$}{\leftarrow}}

% Document
\usepackage{fancyhdr}
  \fancypagestyle{plain}{
    \fancyhf{}
    \fancyhead[L]{\thetitle}
    \fancyhead[R]{\theauthor}
  }

\makeatletter
\newcommand{\subtitle}[1]{\gdef\@subtitle{#1}}
\newcommand{\@subtitle}{}
\makeatother

\makeatletter
\def\@maketitle{
  \newpage 
  \null 
  \vskip 1em 
  \begin{center}
    \let \footnote \thanks 
      {\LARGE \@title \par}
      \vskip 0.5em
      {\large \@subtitle \par}
      \vskip 1em
      {\large \@date}
  \end{center}
  \par 
  \vskip 1em 
}
\makeatother


\title{ASCrypto 2025: MPC and ZKP}
\author{Emanuel Nicolás Herrador}
\newcommand{\speaker}{Sophia Yakoubov}
\date{September 29-30 of 2025}

\begin{document}
  \maketitle 
  \noindent\begin{tabular}{@{}ll}
    Author & \theauthor \\
    Speaker & \speaker
  \end{tabular}

  \section{Example: Prove sudoku solvability}
  We can see it with an example if Alice wants to prove Dani that she knows the 
  solution of a Sudoku problem.

  Here, we're focused in completeness (if the statement is true, Dani accept),
  soundness (if the statement is false, Dani rejects even if Alice cheats) 
  and zero knowledge (Dani learns nothing other than the fact that the statement is true).
  The way to formalize ZK is via the existence of a simulator:
  $\forall \text{ PPT } D^*, \exists \text{ PPT } S : \text{VIEW}(D^*) \equiv S(x)$ where PPT 
  is probabilistic polynomial time and $D^*$ is malicious.

  \paragraph{First try}
  The way to solve it is doing a permutation of the numbers in the sudoku and then send it 
  to Dani.
  He should check the constraints and if the permutation is okay.
  Therefore, we get completeness.
  For soundness, it holds because if Alice could fool Dani then she could solve the sudoku 
  and then the existence of a solution for this sudoku instance holds.
  However, we can't have ZK because Dani can very much reverse engineer Alice solution.

  \paragraph{Second try}
  Alice can send only an specific row with permutated numbers.
  Here, we've completeness and ZK, but not soundness obviously.

  \paragraph{Third try}
  Now, Alice starts sending the sudoku empty and Dani can ask to open a constrain $i$.
  Then, Alice shows the sudoku with this constrain (and the other numbers blocked).
  Therefore, here we've completeness and ZK.
  We can create a simulator and it works with probability $\frac{1}{28}$ where $28$
  is the total of constraints.
  But it doesn't have soundness because exists a chance where Alice cheats and 
  answer correctly to Dani.
  If Alice cheated, she might get away with it with probability less or equal to $\frac{27}{28}$.

  The solution for that is repeating it $k$ times such that $\left(\frac{27}{28}\right)^k$ is small enough.
  With that in mind, we can solve soundness problem without getting away ZK.

  \begin{remark}
    The constraints are revealing a column, row, square or the initial sudoku conditions.
    Each one is made with different permutations, so Dani doesn't learns anything.
  \end{remark}

  \paragraph{How to do this online?}
  Here we can use commitments.
  The properties here are hiding (commit reveals nothing about what's inside without 
  the key) and binding (commit can only be opened to one thing).

  Each property can be perfect (unbreakable even with unlimited resources)
  or computational (reliant on the hardness of some problem).

  To construct it, Alice sends the full sudoku with commitments and when Dani ask for 
  constraint $i$, Alice answer with the key for this constraint letting Dani open it.
  Here, we get completeness, soundness (from binding) and ZK (from hiding).

  \paragraph{In reality}
  In practice we want to prove thinks like identity or possession of credentials,
  correct computation, or more generally knowledge/existence of $w$ such that $R(x,w)=1$.
  A way to solve it is transform a problem to a sudoku that can be solved like that.
  However, the sudoku will be bigger and then it will be inefficient.

  \section{Multi-Party Computation}
  \paragraph{First example}
  Suppose Alice and Dani picks each one a random number between 1 and 10.
  We want privacy, i.e., if $x_A \neq x_D$ that is all they learn.
  A way to do that is closing eyes and open it in the corresponding $x$-th second.
  If they opens it at the same time, then they've the same number.
  Otherwise, not.

  If we've more people, we want correctness and $t$-privacy (the combined views of 
  $t$ or fewer participants reveal nothing other than $y$).

  \section{ZKP from MPC}
  \paragraph{Attempt 1}
  We want to use a MPC protocol that garantice privacy between the two parties.
  So the communication complexity here is $\text{poly}(k\abs{R})$ but using lightweight
  tools (commitments) reducing to sudoku.
  However, if we run 2PC, the complexity is better $O(k\abs{R})$ but with heavyweight
  tools (like public key operations).

  We want to construct MPC from lightweight tools.
  With more participants, we can get $t$-privacy for $t < \frac{n}{2}$ using only 
  lightweight tools.
  However, there are another way using randomness.

  \paragraph{Attempt 2}
  In this attempt Alice runs the MPC protocol in her mind (for 3 parties).
  We're supposing MPC with 1-privacy and perfect correctness.

  The protocol can be seen as the sudoku protocol (with commitments) and where the 
  constraint $i$ is that party $i$ did not cheat and output is $1$.
  Here, the communication between MPC nodes is commitment and also the party choices.

  In this case if Alice runs MPC honestly then all works and we've completeness (that 
  follows form MPC correctness).
  For soundness, we've it because we've perfect correctness (assumption about the 
  MPC protocol) and, therefore, if all of them hold, the statement is true.
  Therefore, to convince Dani, Alice must cheat on behalf of at least one party.
  The probability is $\frac{2}{3}$ and therefore repeating it $k$ times the probability will
  be small enough: $\left(\frac{2}{3}\right)^k$.
  However, here we don't have ZK.
  The ZK can be solved using secret sharing.

  \paragraph{Results of this way to solve the problem}
  Here, if we run MPC in the head, the communication complexity will be $O(k\abs{\text{VIEW}}) = O(k\abs{R})$
  and the tools used is lightweight ones (commitments).
  Therefore, it's better than the another two ways (reducing to sudoku problem 
  or running 2MPC in the two parties Alice and Dani).
\end{document}
