\documentclass{article}
\usepackage[utf8]{inputenc}
\usepackage[english]{babel}
\usepackage{geometry}
  \geometry{
    a4paper,
    total = {170mm, 257mm},
    left = 20mm,
    top = 20mm,
  }
\usepackage{graphicx}
\usepackage{titling}
\usepackage{hyperref}

\setlength{\headheight}{14pt}

\usepackage{amsmath}
\usepackage{amsthm}
\usepackage{amssymb}

% Code environment 
\usepackage{minted}
\usepackage{caption}

\setminted{
  mathescape,
  linenos,
  numbersep=3pt,
  frame=lines,
  framesep=1mm,
  baselinestretch=1,
  fontsize=\footnotesize,
  breaklines=true,
  xleftmargin=1em,
  xrightmargin=0em,
}

\newcounter{code}
\newcommand{\listofcode}{\listof{code}{List of Codes}}
\captionsetup[code]{labelfont=bf, name=Code}

\newcommand{\codefile}[4][python]{%
  \refstepcounter{code}%
  \captionsetup[listing]{name=Code}%
  \captionof{listing}{#2}%
  \label{#3}%
  \inputminted{#1}{#4}%
  }

% Pseudocode environment
\usepackage[ruled,vlined]{algorithm2e}

% Math environments
\usepackage{aliascnt}
\renewcommand{\theequation}{\thesection.\arabic{equation}}

\newenvironment{iproof}[1][Proof idea]
  {
    \begin{proof}[#1]
      \renewcommand{\qedsymbol}{}
  }
  {
  \end{proof}
  }

% Commands 
\usepackage{xparse}

\newcommand{\N}{\mathbb{N}}
\NewDocumentCommand{\R}{g g}{
  \IfNoValueTF{#1}
    {\mathbb{R}}
    {
    \IfNoValueTF{#2}
      {\mathbb{R}^{#1}}
      {\mathbb{R}^{#1 \times #2}}
    }
}

% Document
\usepackage{fancyhdr}
  \fancypagestyle{plain}{
    \fancyhf{}
    \fancyhead[L]{\thetitle}
    \fancyhead[R]{\theauthor}
  }

\makeatletter
\newcommand{\subtitle}[1]{\gdef\@subtitle{#1}}
\newcommand{\@subtitle}{}
\makeatother

\makeatletter
\def\@maketitle{
  \newpage 
  \null 
  \vskip 1em 
  \begin{center}
    \let \footnote \thanks 
      {\LARGE \@title \par}
      \vskip 0.5em
      {\large \@subtitle \par}
      \vskip 1em
      {\large \@date}
  \end{center}
  \par 
  \vskip 1em 
}
\makeatother


\title{CatioCrypto 2025: Zero-Knowledge Proofs}
\author{Emanuel Nicolás Herrador}
\newcommand{\speaker}{Armando Faz Hernandez}
\date{27 Septiembre 2025}

\begin{document}
  \maketitle 
  \noindent\begin{tabular}{@{}ll}
    Autor & \theauthor \\
    Speaker & \speaker
  \end{tabular}

  \section{Commitments Schemes (Esquemas de comprometimiento)}
  \subsection{Definiciones}
  Como analogía, tenemos que pensar en que Ana piensa un número y Beto lo tiene 
  que adivinar.
  Para comprometerse en no cambiar el número luego de la elección de Beto, la idea 
  es colocarlo dentro de una caja con candado y entregárselo a Beto.
  Luego de la elección de Beto, Ana le otorga la llave para verificar si acertó.

  Las propiedades que queremos son:
  \begin{itemize}
    \item Ocultante: Cuando Beto recibe la caja no puede saber el valor hasta que 
      reciba la llave.
    \item Vinculante: Cuando se abra la caja no se puede mostrar un valor diferente
      al almacenado 
  \end{itemize}

  Formalmente, un esquema de comprometimiento no interactivo define los algoritmos:
  \begin{itemize}
    \item $\text{Inicializa}(1^\lambda)$: Retorna parámetros del sistema para un nivel de
      seguridad $\lambda$
    \item $\text{Compromete}(m,r) \to (c,d)$: Toma el mensaje $m$ al que se va a 
      comprometer y una entrada aleatoria $r$, retorna un compromiso $c$ y un valor de 
      apertura $d$ 
    \item $\text{Verifica}(c,m,d) \to \{0,1\}$: Retorna un  bit indicando si la verificación 
      es exitosa
  \end{itemize}

  Las propiedades se traducen en:
  \begin{itemize}
    \item Ocultante: El compromiso $c$ no revela información sobre el mensaje $m$ 
    \item Vinculante: Ningún adversario puede generar $c$, $m\neq m'$ y $d,d'$ tal que 
      ambos acepten $\text{Verifica}(c, m, d)$ y $\text{Verifica}(c, m', d')$.
  \end{itemize}

  Además, estas tienen, a su vez, las siguientes propiedades:
  \begin{itemize}
    \item Incondicional: Aún con poder de cómputo infinito no puede violar la propiedad 
    \item Computacional: A un atacante limitado polinomialmente le es difícil violar la propiedad
  \end{itemize}

  Hay 4 posibilidades pero lo más común es ver ``Incondicional vinculante y computacional ocultante'',
  o ``Computacional vinculante e incondicional ocultante''; mientras que no se cumple que ambas puedan 
  ser incondicionales.

  \subsection{Ejemplo: lanzar una moneda por teléfono}
  Alicia y Beto deciden otorgar un objeto a aquél que gane una tirada pero no confían 
  en el otro.
  El protocolo para el lanzamiento de moneda es:
  \begin{enumerate}
    \item Ana se compromete al valor de un bit aleatorio $b_A$
    \item Ana envía su compromiso $C$ a Beto
    \item Beto escoge un bit $b_B$ aleatoriamente y lo envía a Alicia 
    \item Alicia abre $C$ para que Beto conozca $b_A$ 
    \item Ambos calculan el resultado $b = b_A \oplus b_B$
  \end{enumerate}

  Con esto se asegura el lanzamiento de la moneda y el protocolo completo solo 
  queda en agregar que Beto adivine.
  Aquí $b$ sería el lanzamiento.

  \paragraph{Comprometer un bit con RSA}
  Un esquema de compromiso para un bit se puede hacer usando RSA.
  \begin{itemize}
    \item Ana tiene un par de llaves RSA: pública $(e,n)$ y privada $(d)$
    \item Ana se compromete a un bit $b$, escoge $x_b \leftarrow \{0,\dots,n\}$ 
      aleatoriamente tal que su bit menos significativo es $b$ 
    \item Ana envía $C = x_b^e \pmod{n}$ a Beto 
  \end{itemize}
  Tiene las siguientes propiedades:
  \begin{itemize}
    \item Ocultante: Notar que Beto no puede saber $x_b$ a menos que pueda romper RSA.
    \item Vinculante: Una vez que escogió el bit, no lo puede cambiar dada la construcción del compromiso.
  \end{itemize}

  Otra forma de hacerlo es usando $QR_N$ (residuos cuadráticos módulo $N$), el cual 
  únicamente usa $N = pq$ con $p,q$ primos aleatorios.
  La forma es:
  \begin{enumerate}
    \item Ana elige aleatoriamente un módulo $N = pq$ y envía $(N,x)$ a Beto con 
      \begin{equation*}
        x \overset{\$}{\leftarrow} \begin{cases} QR_N &\text{ si }b=0 \\ \Z/QR_N &\text{ si }b=1 \end{cases}
      \end{equation*}
    \item Ana abre el compromiso enviando $(p,q)$
    \item Beto verifica que $N = pq$ y que si $x$ es $QR_N$ entonces retorna $b=0$,
      pero sino retorna $b=1$.
  \end{enumerate}

  \subsection{Otros protocolos}
  \subsubsection{Pedersen}
  Sea $G = \gen{g}$ un grupo cíclico de orden $q$ y sean $g,h\in G$ dos generadores.
  El compromiso de Pedersen para escalares $m\in\Z/q\Z$ es:
  \begin{enumerate}
    \item Ana obtiene $r \getrandom \Z/q\Z$ y calcula $c=g^mh^r$
    \item Ana abre el compromiso enviando $(m,r)$ a Beto
    \item Si beto verifica $c = g^mh^r$ entonces retorna $m$, sino error.
  \end{enumerate}

  \subsubsection{ElGamal}
  Sea $G = \gen{g}$ un grupo cíclico de orden $q$ y sean $g,h\in G$ dos generadores.
  El compromiso de ElGamal para elementos $M\in G$ es:
  \begin{enumerate}
    \item Ana obtiene $r \getrandom \Z/q\Z$ y calcula $c_0 = g^r,c_1 = Mh^r$. El compromiso es $(c_0,c_1)$.
    \item Ana abre el compromiso enviando $(M,r)$ a Beto
    \item Si beto verifica $c_0 = g^r,c_1 = Mh^r$ entonces retorna $M$, sino error.
  \end{enumerate}

  \section{Pruebas}
  La idea es que se desea brindar acceso a un recurso solo a ciertos usuarios, los cuales se
  pueden identificar con una contraseña.
  Para ello, deben enviar su contraseña al servidor para verificar su validez pero el problema 
  es que cualquiera escuchando al medio puede saberla.
  Las pruebas entran para remediar esta situación.

  Para esto, el servidor solo necesita saber si el usuario sabe su contraseña.
  Es decir, debe saber solo un bit y nada más.
  Para ello se pretende un protocolo donde al final de la interacción entre el 
  usuario y el servidor, este último obtiene un bit que indica si el usuario pudo 
  identificarse exitosamente.

  \begin{definition}[Zero-Knowledge Protocol]
    Un protocolo es de conocimiento nulo si comunica exactamente el conocimiento 
    previsto y ningún (cero) conocimiento adicional.
  \end{definition}

  Estos protocolos tienen dos partes (prover/probador y verifier/verificador), los cuales 
  vamos a llamar como Perla y como Víctor en los ejemplos.
  El prover quiere convencer al verifier sobre la verdad de alguna declaración/statement,
  mientras que el verifier quiere verificar que el statement sea verdadero.

  \subsection{Construcción de un ZK Protocol}
  Supongamos que tenemos un criptosistema de llave pública y que el usuario $A$ posee 
  una public key $P_A$ y una private key $k_A$.
  El statement es ``Perla quiere probar que ella es el usuario $A$''.
  Con ello, el protocolo es:
  \begin{enumerate}
    \item Víctor escoge un mensaje $M$ aleatoriamente y envía $C = E(M, P_A)$ a Perla 
    \item Perla descifra $C$ usando $k_A$ y envía el resultado $M'$ a Víctor
    \item Víctor acepta la declaración si $M' = M$
  \end{enumerate}
  \begin{remark}
    No es Zero-Knowledge porque el atacante puede actuar como el Verifier 
    y hacer que el Prover le descifre mensajes cifrados de los cuales no conoce 
    el texto original.
    El prover no tiene forma de evitar esta situación con este protocolo.
  \end{remark}

  En este caso, cuando el Verifier envía $C$, no existen garantías de que conozca
  también $M$.
  Para solucionarlo, se debe agregar que el Prover envíe también el compromiso de $M$.
  El esquema final es:
  \begin{enumerate}
    \item Víctor escoge un mensaje $M$ aleatoriamente y envía $C = E(M,P_A)$ a Perla 
    \item Perla descifra $C$ usando $k_A$ y envía un compromiso $c$ de $M'$ a Víctor 
    \item Víctor envía $M$ a Perla 
    \item Perla revisa si $M = M'$ y retorna error sino se cumple.
      Caso contrario, abre el compromiso enviando $(r,M)$ a Víctor 
    \item Víctor acepta la declaración si $M' = M$ y si el compromiso para la
      verificación
  \end{enumerate}
  \begin{remark}
    Con esto ya se llega a que es un protocolo de conocimiento nulo.
  \end{remark}

  \subsection{Sistema de prueba interactivo}
  Ahora los protocolos se llevan a cabo como interacciones entre dos máquinas
  interactivas de Turing.
  El Prover tiene poder computacional infinito mientras que el Verifier tiene 
  un tiempo de cómputo limitado polinómicamente.
  El par $(P,V)$ recibe una entrada $x$ y al final de la interacción devuelve si 
  se acepta o rechaza respecto a si pertenece al lenguaje $L \subseteq \{0,1\}^*$
  (i.e., si se verifica $x$ o no).

  El par $(P,V)$ es un sistema de prueba interactivo para $L$ si satisface:
  \begin{itemize}
    \item Completitud: Si $x\in L$ entonces la probabilidad de que $(P,V)$ rechace $x$
      es despreciable en la longitud de $x$ 
    \item Solidez (Soundness): Si $x\notin L$, entonces para cualquier prover $P^*$, la
      probabilidad de que $(P^*, V)$ acepte $x$ es despreciable en la longitud de $x$.
  \end{itemize}

  \paragraph{Argumentos interactivos}
  Un argumento interactivo para un lenguaje $L$ es similar a un sistema de prueba 
  interactivo, excepto que:
  \begin{itemize}
    \item El prover tiene tiempo de cómputo limitado polinómicamente y es probabilístico 
    \item En la completitud se requiere que para cada $x \in L$ exista una entrada auxiliar 
      que permita al probador convencer al verificador 
    \item En la solidez, se cambia por cualquier prover que tenga tiempo de cómputo 
      limitado polinomialmente y que sea probabilístico
  \end{itemize}

  \subsection{Prueba de conocimiento (Proof of Knowledge)}
  En una prueba de conocimiento el prover afirma saber una pieza de información.
  Vamos aquí a tener un algoritmo que va a ser el extractor quien interactúa con el 
  prover (posiblemente malicioso) para saber verificar si realmente conoce esta 
  información.

  Un ejemplo es ``Schnorr Knowledge Proof'' donde el statement es 
  ``Sé el logaritmo discreto $x = \log_g(y)$, i.e., $y = g^x$''.

  \todo{Agregar contenido completo del protocolo de Schnorr, pruebas de conocimiento 
  nulo, Fiat-Shamir y firma digital con Schnorr en base a las filminas.}
\end{document}
