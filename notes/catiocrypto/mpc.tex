\documentclass{article}
\usepackage[utf8]{inputenc}
\usepackage[spanish]{babel}
\usepackage{geometry}
  \geometry{
    a4paper,
    total = {170mm, 257mm},
    left = 20mm,
    top = 20mm,
  }
\usepackage{graphicx}
\usepackage{titling}
\usepackage{hyperref}

\setlength{\headheight}{14pt}

\usepackage{amsmath}
\usepackage{amsthm}
\usepackage{amssymb}

% Draw
\usepackage{tikz}
\usetikzlibrary{positioning, arrows.meta, shapes.geometric, calc}
\tikzset{
  >={Stealth[length=3mm]},
  box/.style={draw, rounded corners, fill=blue!10, inner sep=6pt, text width=5cm, align=center, minimum height=3cm},
  data/.style={draw, ellipse, fill=green!10, inner sep=4pt, align=center},
}

% Code environment 
\usepackage{minted}
\usepackage{caption}

\setminted{
  mathescape,
  linenos,
  numbersep=3pt,
  frame=lines,
  framesep=1mm,
  baselinestretch=1,
  fontsize=\footnotesize,
  breaklines=true,
  xleftmargin=1em,
  xrightmargin=0em,
}

\newcounter{code}
\newcommand{\listofcode}{\listof{code}{List of Codes}}
\captionsetup[code]{labelfont=bf, name=Code}

\newcommand{\codefile}[4][python]{%
  \refstepcounter{code}%
  \captionsetup[listing]{name=Code}%
  \captionof{listing}{#2}%
  \label{#3}%
  \inputminted{#1}{#4}%
  }

% Pseudocode environment
\usepackage[ruled,vlined]{algorithm2e}

% Math environments
\usepackage{aliascnt}
\renewcommand{\theequation}{\thesection.\arabic{equation}}

\theoremstyle{definition}
\newtheorem{definition}{Definición}[section]
\providecommand*{\definitionautorefname}{Definición}

\theoremstyle{remark}
\newtheorem*{remark}{Nota}

\theoremstyle{plain}
\newtheorem{theorem}{Teorema}[section]

\theoremstyle{plain}
\newtheorem{lemma}{Lema}[section]
\providecommand*{\lemmaautorefname}{Lema}

\theoremstyle{plain}
\newaliascnt{proposition}{theorem}
\newtheorem{proposition}[proposition]{Proposición}
\aliascntresetthe{proposition}
\providecommand*{\propositionautorefname}{Proposición}

\theoremstyle{plain}
\newtheorem{corollary}{Corolario}[theorem]

\newenvironment{iproof}[1][Proof idea]
  {
    \begin{proof}[#1]
      \renewcommand{\qedsymbol}{}
  }
  {
  \end{proof}
  }

% Commands 
\usepackage{xparse}

\usepackage{xcolor}
\newcommand{\todo}[1]{
  \begin{quotation}
    \textcolor{red}{\textbf{TODO: }#1}
  \end{quotation}
}

\newcommand{\N}{\mathbb{N}}
\NewDocumentCommand{\R}{g g}{
  \IfNoValueTF{#1}
    {\mathbb{R}}
    {
    \IfNoValueTF{#2}
      {\mathbb{R}^{#1}}
      {\mathbb{R}^{#1 \times #2}}
    }
}

\newcommand{\Z}{\mathbb{Z}}
\newcommand{\F}{\mathbb{F}}

\newcommand{\abs}[1]{\left| #1 \right|}
\newcommand{\gen}[1]{\left\langle #1 \right\rangle}

\newcommand{\getrandom}{\overset{\$}{\leftarrow}}

% Document
\usepackage{fancyhdr}
  \fancypagestyle{plain}{
    \fancyhf{}
    \fancyhead[L]{\thetitle}
    \fancyhead[R]{\theauthor}
  }

\makeatletter
\newcommand{\subtitle}[1]{\gdef\@subtitle{#1}}
\newcommand{\@subtitle}{}
\makeatother

\makeatletter
\def\@maketitle{
  \newpage 
  \null 
  \vskip 1em 
  \begin{center}
    \let \footnote \thanks 
      {\LARGE \@title \par}
      \vskip 0.5em
      {\large \@subtitle \par}
      \vskip 1em
      {\large \@date}
  \end{center}
  \par 
  \vskip 1em 
}
\makeatother


\title{CatioCrypto 2025: Multi-Party Computation}
\author{Emanuel Nicolás Herrador}
\newcommand{\speaker}{Eduardo Soria-Vázquez}
\date{26-27 Septiembre 2025}

\begin{document}
  \maketitle 
  \noindent\begin{tabular}{@{}ll}
    Autor & \theauthor \\
    Speaker & \speaker
  \end{tabular}

  \section{Introducción}
  Algunos \underline{ejemplos} acerca de qué se ocupa la criptografía son:
  \begin{itemize}
    \item \textit{Verifiable Computation}: Demostrar que se ha ejecutado un programa,
      sin que quien lo verifique tenga que re-ejecutarlo.
    \item \textit{Zero-Knowledge Proofs}: Similar a VC, pero quien demuestra 
      hacer el cálculo pueden mantener algunos datos ocultos.
    \item \textit{Secret Sharing}: Fragmentar un secreto entre varias entidades,
      de tal forma que solo pueda reconstruirse si ciertos subconjuntos
      de ellas acuerdan hacerlo.
    \item \textit{Multi-Party Computation} (MPC): Calcular sobre datos ``sin verlos'',
      de forma distribuida.
    \item Otros: privacidad diferencial (DP), recuperación privada de información (PIR),
      cifrado totalmente homomórfico (FHE), cifrado funcional (FE), time-lock puzzles,
      blockchaings, criptomonedas, ...
  \end{itemize}

  \section{Multi-Party Computation}
  \subsection{Protocolo MPC}
  Un protocolo MPC consta de $n$ entidades que desconfían las unas de las otras y de modo
  que cada una pueda aportar un valor $P_i$ secreto.
  La idea/objetivo es calcular un función $f$ aplicada a estos valores en conjunto pero sin mostrarlos 
  a otra de las entidades.

  Una forma de resolverlo sin MPC es con una tercera parte de confianza, pero el 
  problema es que no siempre existe.
  Lo que vuelve difícil a MPC es la presencia de un adversario 
  que participa y corrompe a ciertas entidades para coordinar su ataque:
  \begin{itemize}
    \item Corrupción pasiva: las entidades corruptas hacen lo que deben pero comparten 
      información entre ellas para tratar de aprender más cosas sobre las demás entidades 
    \item Corrupción activa: las entidades corruptas pueden actuar de forma arbitraria 
      para sabotear los objetivos de seguridad
  \end{itemize}
  El \underline{objetivo} es lograr todas las propiedades que lograríamos con una tercera 
  parte de confianza pero sin ella.

  \subsection{Primitivas}
  Algunas de las prmitivas usadas por MPC son:
  \begin{itemize}
    \item Garbled Circuits
    \item Secret Sharing 
    \item Oblivious Transfer 
    \item Homomorphic Encryption
  \end{itemize}

  \subsubsection{Secret Sharing (compartición de secretos)}
  \begin{definition}[Esquema de compartición de secretos]
    Sea $\F$ un cuerpo finito y $n,t\in \Z : 0\leq t < n$.
    Sean las entidades $P_1,\dots,P_n$ y un compartidor en posesión de un 
    secreto $s\in\F$.
    Un esquema $(n,t)$-umbral de compartición de secretos consta de las siguientes fases:
    \begin{enumerate}
      \item Fase de compartición: El Compartidor fragmenta $s$ en $n$ partes $s_1,\dots,s_n\in\F$
        y la entidad $P_i$ recibe su parte $s_i\in\F$ 
      \item Fase de reconstrucción: Al menos $t+1$ entidades envían su parte a quien quieran 
        que reconstruya $s$.
    \end{enumerate}

    Y tal que satisface las siguientes propiedades:
    \begin{itemize}
      \item $t$-privacidad: Cualquier conjunto de a lo sumo $t$ partes no revela ninguna 
        información sobre el secreto $s\in\F$
      \item $(t+1)$-reconstrucción: Cualquier conjunto de $t+1$ partes determina de forma 
        única el secreto $s\in\F$
    \end{itemize}
  \end{definition}
  \begin{definition}[Esquema lineal de compartición de secretos]
    Decimos que el esquema es \textit{lineal} si además satisface que $\forall s,t\in\F$
    compartidos como $(s_1,\dots,s_n),(t_1,\dots,t_n)\in\F^n$, tenemos que 
    $(s_1+t_1,\dots,s_n+t_n)$ es una compartición de $s+t\in\F$.
  \end{definition}
  \begin{remark}
    Abusaremos notación y \underline{nos restringiremos} a esquemas de compartición de secretos 
    que sean sobre un cuerpo finito $\F$, umbrales y lineales (sobre $\F$).
    Además, usaremos la notación definida como $[s] := (s; s_1,\dots,s_n)$, de modo que 
    $[s] + [t] = [s+t]$.
  \end{remark}

  \paragraph{Compartición de secretos aditiva}
  Sea $F_p := (\Z/p\Z, +, \cdot),\ \Z/p\Z = (0,\dots,p-1),\ p$ nro. primo, el esquema es:
  \begin{enumerate}
    \item Fase de compartición:
      \begin{enumerate}
        \item Compartidor escoge de forma uniformemente aleatoria $s_1,\dots,s_n\in\F_p : s_1+\dots+s_n=s$.
        \item Compartidor envía la parte $s_i$ a $P_i$
      \end{enumerate}
    \item Fase de reconstrucción:
      \begin{enumerate}
        \item Cada $P_i$ envía $s_i$ a la entidad que quieren que reconstruya el secreto 
        \item Dicha entidad calcula $s=s_1+\dots+s_n$
      \end{enumerate}
  \end{enumerate}
  \begin{remark}
    Notar que el umbral $t$ debe ser $n-1$ para que sea reconstruible teniendo todos los 
    pedacitos de secretos.
    El hecho de que no sea reconstruible con $n-1$ entidades se reduce al OTP porque si 
    tenemos $s_1,\dots,s_{n-1}\in\F_p$ elegidos uniformemente aleatorios, entonces 
    $s_n = s - (s_1+\dots+s_{n-1})$ es también uniformemente aleatorio.
  \end{remark}

  Las \textit{ventajas} de la compartición de secretos aditiva es que tiene un tamaño 
  óptimo de las partes (cada uno es tan grande como el secreto), por lo cual es 
  muy eficiente a nivel computacional.
  Sin embargo, algunas \textit{contras} son que el umbral es siempre fijo ($t = n-1$,
  por lo que no podemos elegir),
  que no tiene propiedades multiplicativas y que tampoco ofrece resistencia ante 
  adversarios activos.

  \paragraph{Compartición de secretos replicada (con 3 entidades)}
  Se puede generalizar con $n > 3$ pero se suele usar de este modo.
  El esquema es:
  \begin{enumerate}
    \item Fase de compartición:
      \begin{enumerate}
        \item Compartidor escoge de forma uniformemente aleatoria $s_1,s_2,s_3\in\F_p : s_1+s_2+s_3=s$.
        \item Compartidor envía las partes $s_{i-1},s_{i+1}$ a $P_i$
      \end{enumerate}
    \item Fase de reconstrucción:
      \begin{enumerate}
        \item Cada $P_i$ envía $s_{i-1}$ y $s_{i+1}$ a la entidad que quieren que reconstruya el secreto 
        \item Dicha entidad calcula $s=s_1+s_2+s_3$
      \end{enumerate}
  \end{enumerate}
  \begin{remark}
    El umbral $t$ es $1$ porque es $1$-privado y $2$-reconstruible.
  \end{remark}

  Las \textit{ventajas} son que:
  \begin{itemize}
    \item Tiene propiedades multiplicativas 
    \item Ante un adversario pasivo, la reconstrucción puede requerir de menos comunicación 
      que la compartición de secretos aditiva (por la redundancia).
      Como ejemplo, si es alguien interno (de las $3$ entidades), entonces solo requiere de 
      un valor, lo cual es una mejora en el coste de comunicación respecto al esquema lineal.
      Si la entidad es externa, se mantiene.
    \item Tiene propiedades de detección de errores.
      Ante un adversario activo, se puede detectar si una entidad se desvía del protocolo.
      En el caso de que no envíe nada, por la redundancia se salva.
      Pero si envía datos incorrectos, podemos detectarlo y frenar el cálculo, aunque no 
      reconstruir.
  \end{itemize}
  En cuanto a los \textit{inconvenientes}, el principal es que el tamaño de las partes es 
  superior al del secreto y crece rapidísimo en la  generalización:
  $\binom{n}{t}$ elementos de $\F_p$ por entidad.
  Notar que en el caso $t=1,n=3$, el tamaño de las partes es el doble que del secreto.

  \paragraph{Compartición de secretos de Shamir ($n$ entidades)}
  Sea $F_p := (\Z/p\Z, +, \cdot),\ \Z/p\Z = (0,\dots,p-1),\ p > n$ nro. primo, el esquema es:
  \begin{enumerate}
    \item Fase de compartición:
      \begin{enumerate}
        \item Compartidor escoge de forma uniformemente aleatoria $f \in \F_p[X]$ de grado $t$ tal que $f(0) = s$
        \item Compartidor envía a cada $P_i$ su parte $f(i) = s_i$
      \end{enumerate}
    \item Fase de reconstrucción:
      \begin{enumerate}
        \item Cada $P_i$ envía $s_i$ a la entidad que quieren que reconstruya el secreto 
        \item Dicha entidad obtiene $f \in \F_p[X]$ mediante interpolación y recupera el secreto 
          evaluando $f(0) = s$
      \end{enumerate}
  \end{enumerate}
  \begin{remark}
    El umbral $t$ es el grado del polinomio, dado que con $t$ puntos no somos capaces de determinar el 
    polinomio de grado $t$ con interpolación, mientras que con $t+1$ sí se puede.
    Además, esta compartición es \textit{lineal}.
  \end{remark}

  Se puede usar la Matriz de Vandermonde para demostrar linearidad (incluso isomorfismo),
  $t$-privacidad y $(t+1)$-reconstruibilidad.
  Recordemos que una matriz de Vandermonde es:
  \begin{equation*}
    Van^{u\times v}(a_1,\dots,a_u) = \begin{bmatrix}
      a_1^0 & \dots & a_1^{v-1} \\ 
      \vdots & \vdots & \vdots \\ 
      a_u^0 & \dots & a_u^{v-1}
    \end{bmatrix}
  \end{equation*}
  Tal que si $u = v$ y $\forall i\neq j, a_i \neq a_j$ entonces $Van^{u \times v}(a_1,\dots,a_u)$
  es invertible.
  Además lo son también todas sus submatrices cuadradas.
  Con ello, se puede ver que sea $f(X) = \sum_{i=0}^t f_i \cdot X^i$ el polinomio en $\F_p[X]$ de grado $\leq t$,
  entonces $(f(a_1),\dots,f(a_{t+1})) = Van^{u\times v}(a_1,\dots,a_{t+1})\cdot (f_0,\dots,f_t)$.

  Las \textit{ventajas} de este esquema de compartición son que el tamaño de cada parte
  es igual al tamaño del secreto, tiene propiedades multiplicativas y es equivalente 
  a un código de Reed Solomon (por lo que en presencia de un adversario activo se pueden 
  explotar las propiedades de detección y corrección de errores).
  En cuanto a las \textit{contras}, esta se basa principalmente en que la reconstrucción es 
  algo costosa respecto a los otros dos esquemas vistos, y de que el tamaño del cuerpo tiene 
  que ser mayor que el grado (sino la matriz de Vandermonde dejaría de ser invertible).

  \subsection{Propiedades y características}
  Las métricas de eficiencia para los protocolos de MPC son:
  \begin{itemize}
    \item Cantidad de datos que se comunican (complejidad de comunicación).
      Suele ser el cuello de botella de los protocolos y es lo que se busca bajar.
    \item Número de rondas de comunicación (complejidad de rondas) 
    \item Complejidad computacional
  \end{itemize}
  La combinación de las distintas dimensiones determina la eficiencia que es posible (en ocasiones,
  tenemos cotas inferiores) o directamente si es posible o imposible hacer MPC con dicha selección.
  Veamos cada un de las dimensiones que puede tener el protocolo.

  \paragraph{Modelo de computación}
  Tenemos distintas formas de representar la computación.
  Podemos considerar circuitos booleanos (generalmente considerando and y or), circuitos aritméticos 
  (sobre cuerpos o anillos finitos) o memoria de acceso aleatorio (RAM).
  Cualquier función o programa se puede expresar en cualquiera de los tres modelos.

  \paragraph{Fases en el protocolo}
  Podemos considerar las siguientes posibilidades:
  \begin{itemize}
    \item Paradigma ``Offline-Online'': Existe una fase de preprocesado (``offline'') en la cual no es 
      necesario que las entidades hayan decidido sus inputs para la función.
      La fase ``online'' comienza cuando las entidades conocen sus inputs, y generalmente esta puede 
      ejecutarse más rápido debido al trabajo avanzado en el preprocesado.
      Hay dos tipos principales de preprocesados: dependiente o independiente de la función.
      En el primer caso se asume el conocimiento de la función que se calculará en la fase online mientras
      que en la segunda no (en realidad se sabe algo pero es una cota).
    \item Sin preprocesado: existe una única fase, por lo que no se avanza antes de conocer los inputs.
  \end{itemize}

  \paragraph{Suposiciones sobre la red}
  Podemos hacer dos tipos de suposiciones sobre la red del MPC:
  \begin{itemize}
    \item Síncrona: Las entidades están de acuerdo en la misma hora actual y se establecen 
      momentos límite para la recepción de mensajes.
      Existe la noción de ``rondas'' de comunicación, por lo que si después de $t$ tiempo no 
      llegan los mensajes entonces se supone que no los envió.
    \item Asíncrona
  \end{itemize}

  \paragraph{Proporción de entidades corruptas}
  \begin{itemize}
    \item Mayoría deshonesta: Al menos una entidad es honesta, es decir $t < n$
    \item Mayoría honesta: Al menos una mayoría (estricta) de las entidades son honestas,
      i.e. $t < \frac{n}{2}$
    \item Mayoría honesta $\frac{2}{3}$: Al menos dos tercios de las entidades son honestas, i.e., $t < \frac{n}{3}$
  \end{itemize}

  \paragraph{Capacidades del adversario}
  Es lo que vimos anteriormente:
  \begin{itemize}
    \item Seguridad pasiva 
    \item Seguridad activa
  \end{itemize}

  \paragraph{Movilidad del adversario}
  \begin{itemize}
    \item Seguridad estática: El adversario especifica qué entidades van a ser corrompidas antes de comenzar 
      la ejecución del protocolo de MPC.
    \item Seguridad adaptativa: El adversario puede decidir qué entidades corromper o controlar de forma adaptativa 
      durante la ejecución del protocolo.
  \end{itemize}

  \paragraph{Capacidad computacional del adversario}
  \begin{itemize}
    \item Seguridad incondicional: La seguridad se basa en teoría de la información y estadística.
      No se impone ninguna restricción en cuanto a la cantidad de recursos computacionales del adversario.
      Imposible de obtener cuando la mayoría es deshonesta (es un teorema).
      Se divide en seguridad estadística (probabilidad despreciable) o perfecta.
    \item Seguridad computacional: El adversario es modelado como una máquina de Turing probabilística en tiempo polinomial
      en vez de tener una capacidad limitada. Es decir, la seguridad reposa sobre la hipótesis de que algún problema 
      específico es difícil de resolver en tiempo polinomial.
  \end{itemize}

  \paragraph{Garantías sobre el output}
  Hay tres posibilidades:
  \begin{itemize}
    \item Entrega garantizada: las entidades honestas siempre reciben el output 
    \item Justo: si las entidades corruptas reciben el output, entonces también lo reciben las honestas.
      El adversario puede decidir si todos tienen el output o ninguno.
    \item Seguridad con aborto: las entidades corruptas pueden recibir el output mientras impiden que las honestas lo hagan.
      Pueden parar el protocolo cuando quieran.
      Aquí puede darse la situación en las que las entidades corruptas aprendan el output pero las demás no.
  \end{itemize}

  \vspace{2em}
  Con ello, es sabido cuáles combinaciones de las dimensiones son factibles o no de realizar.
  \todo{Agregar el cuadro de MPC de las filminas}

  \subsection{MPC a partir de compartición de secretos en circuitos aritméticos}
  Para calcular una función, lo que se hace s:
  \begin{enumerate}
    \item Cada entidad comparte sus inputs secretos correspondientes 
      Sucesivamente calculamos los valores intermedios y outputs del circuito, en forma
      de secreto compartido.
      Las operaciones lineales no requieren de interacción si el esquema de compartición de
      secretos es lineal.
      Productos u operaciones no lineales requieren (casi siempre) interacción.
    \item Reconstruimos los outputs a las entidades que les correspondan dichos resultados.
  \end{enumerate}

  \paragraph{Compartición de secretos de Shamir: Muliplicación}
  Sean $[x]_t, [y]_t$ dos secretos compartidos mediante Shamir para umbral $t$, entonces si cada $P_i$
  calcula localmente $x_i \cdot y_i$, las entidades obtienen \underline{de forma segura} el secreto compartido $[x \cdot y]_{2t}$.
  \begin{remark}
    Lo complejo es que si el grado del polinomio supera $n$ entonces ya no es posible realizar la reconstrucción.
    Por esto tenemos que tener muy presente que el grado de los secretos jamás supere $n$.
  \end{remark}

  \paragraph{MPC en el estilo de BGW, seguridad pasiva}
  Estamos considerando circuitos aritméticos (modelo de computación), red síncrona, sin preprocesado y con el
  adversario pasivo y estático, vamos a alcanzar seguridad perfecta con una mayoría honesta ($t < \frac{n}{2}$).
  La idea general es la vista antes para protocolos de compartición de secretos con multiplicación.

  El protocolo de multiplicación considera que sean $[x]_t,[y]_t$ dos valores a multiplicar entonces:
  \begin{enumerate}
    \item Las entidades calculan de forma local $[z]_{2t} = [x \cdot y]_{2t}$.
      Denotemos las partes del secreto como $(z_1,\dots,z_n)$.
    \item Cada entidad $P_i$ usa el esquema de Shamir para compartir su parte $[z_i]_t$ con el resto de entidades.
    \item De forma local, las entidades calculan $[x\cdot y]_t = \sum_{i=1}^{2t+1} \lambda_i \cdot [z_i]_t$,
      donde $\lambda_i$ son los coeficientes de interpolación de Lagrange.
  \end{enumerate}
  \begin{remark}
    Se impone $t < \frac{n}{2}$ para que el grado del polinomio $[z]_{2t}$ no exceda $n$ dado que sino no tendríamos 
    la suficiente cantidad de puntos para interpolar.
    El protocolo tampoco es seguro contra un adversario activo.
  \end{remark}

  \paragraph{MPC en el estilo de BGW, seguridad activa}
  Es un protocolo muy parecido al anterior, solo que tenemos un preprocesado (independiente de $f$, únicamente con
  cota superior de cantidad de multiplicaciones que hay), adversario activo, seguridad con aborto (podría ser mejor y
  obtenerse output garantizado), y con un umbral de corrupción $t < \frac{n}{3}$.
  Las herramientas que se usan son la idea general para protocolos basados en compartición de secretos, compartición de 
  secretos de Shamir y pares de reducción de grado.

  En el esquema del protocolo el adversario puede comportarse mal de tres formas distintas:
  \begin{itemize}
    \item En la fase de compartición:
      \begin{itemize}
        \item No compartir su secreto: no se puede hacer nada
        \item Mentir y enviar un input incorrecto: no se puede hacer nada pero a lo sumo se puede crear un circuito que verifique
          si el input satisface las condiciones correctas (como estar en cierto rango)
        \item Enviar un secreto como un polinomio de grado distinto de $t$: nos vamos a concentrar en esto.
          Vamos a usar el preprocesado para esto.
      \end{itemize}
    \item En la fase de cálculo, el inconveniente viene en donde nos encontramos con una puerta de multiplicación (o no lineal)
      que requiera interacción y donde podamos mentir (o al final con el output).
      En este caso, como vimos antes, el protocolo de multiplicación no es seguro contra un adversario activo.
    \item Fase de reconstrucción de los outputs: el mismo, mentir.
  \end{itemize}

  Podemos pensar los esquemas de Shamir como algoritmos de detección de errores de Reed Solomon, por lo que podremos ser capaces de
  identificar el error y la entidad corrupta.
  Veamos cómo solucionar cada parte:
  \begin{itemize}
    \item Protocolo de repartición de inputs: Consideramos lo siguiente
      \begin{itemize}
        \item Preprocesado: un secreto correctamente compartido $[r]_t$ donde $r$ es uniformemente aleatorio y 
          únicamente conocido por $P_i$
        \item Inputs: $x$, el input secreto de $P_i$ que nos queremos asegurar de que se comparte correctamente 
        \item Outputs: $[x]_t$
      \end{itemize}
      Luego, vamos a hacer:
      \begin{enumerate}
        \item $P_i$ hace broadcast del valor $e = x - r$ (un canal de broadcast se asegura de que todas las entidades 
          reciben el mismo mensaje, se puede hacer simplemente con que todas las entidades hagan echo)
        \item De forma local, las entidades calculan $[x]_t = e + [r]_t$
      \end{enumerate}
    \item Protocolo de multiplicación (double random share): Consideramos el preprocesado dado por
      \begin{itemize}
        \item Preprocesado: un par de reducción de grado $([r]_t,[r]_{2t})$ donde $r$ es uniformemente aleatorio 
        \item Inputs: $[x]_t, [y]_t$ valores a multiplicar 
        \item Outputs: $[x \cdot y]_t$
      \end{itemize}
      Luego, vamos a hacer:
      \begin{enumerate}
        \item Las entidades calculan de forma local $[x \cdot y]_{2t}$
        \item También de forma local, calculan $[z]_{2t} = [x \cdot y]_{2t} - [r]_{2t}$
        \item Las entidades reconstruyen públicamente $z$ (canal de broadcast), aplicando detección de errores 
        \item De forma local, las entidades calculan $[x \cdot y]_t = z + [r]_t$
      \end{enumerate}
      \begin{remark}
        Para el paso $2$, notemos que $r$ es desconocido por todos y se debe usar una sola vez (sin volverlo a repetir).
        En cuanto al paso $3$, podemos ver $[z]_d$ como una palabra en un Código de Reed-Solomon $[n,d+1,n-d]$.
        Por ello, cuando $t < \frac{n}{3}$ entonces se pueden detectar errores con $d \leq 2t$ y corregir con $d \leq t$.

        La analogía entre Shamir y Reed-Solomon se puede ver como:

        \begin{center}
          \begin{tabular}{|c|c|c|}
            \hline
            Variable & Shamir & Reed-Solomon \\ 
            \hline
            n & Cantidad de entidades & Longitud del código \\ 
            t & Cantidad de corruptos & Cantidad de errores \\ 
            d & Grado del polinomio & Dimensión - 1 \\ 
              & & \\ 
            $t \geq n-d$ & Inseguro & No hay detección de errores \\
            $t < n-d$ & ``Seguridad con aborto'' & Detección de errores \\
            $2t < n-d$ & Entrega garantizada & Existen algoritmos eficientes de corrección de errores \\
            \hline
          \end{tabular}
        \end{center}

        En la práctica, lo que se trata de hacer es una reducción de jugadores sucesivamente.
        A diferencia del código de corrección de errores, la interactividad nos permite mayor posibilidad de detección.
      \end{remark}
  \end{itemize}

  \paragraph{Protocolo de multiplicación (compartición aditiva)}
  El protocolo de multiplicación, si tenemos mayoría deshonesta, se puede implementar usando compartición de secretos 
  aditiva.
  Para ello, digamos que el secreto es nota como $\gen{\cdot}$.
  Luego, consideramos el preprocesado dado por:
  \begin{itemize}
    \item Preprocesado: $\gen{a},\gen{b}, \gen{c}$ con $c = a\cdot b,\ a,b \in \F$ (desconocidos)
    \item Input: $\gen{x}, \gen{y}, \gen{\cdot}$ compartición de secretos aditiva
  \end{itemize}
  Luego la idea es tomar:
  \begin{equation*}
    \begin{aligned}
      x \cdot y &= (x + a - a)(y + b - b) = (\varepsilon - a)(\rho - b) 
                &= \varepsilon\rho - a\rho - \varepsilon b + ab
    \end{aligned}
  \end{equation*}
  donde $\epsilon = x + a, \rho = y + b$.
  Entonces, podemos tomar para la parte del cálculo secreto a: $\varepsilon \rho - \gen{a}\rho - \varepsilon \gen{b} + \gen{c}$
  donde $\varepsilon, \rho$ pueden publicarse.
  Con ello, el esquema sería:
  \begin{enumerate}
    \item Calcular $\gen{\varepsilon} = \gen{x} + \gen{a}$ y $\gen{\rho} = \gen{y} + \gen{b}$
    \item Reconstruimos $\varepsilon, \rho$
    \item Calculamos $\gen{x \cdot y} = \varepsilon\rho - \gen{a}\rho - \varepsilon\gen{b} + \gen{c}$ localmente.
  \end{enumerate}
\end{document}
