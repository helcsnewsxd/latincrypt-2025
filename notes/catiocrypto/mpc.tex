\documentclass{article}
\usepackage[utf8]{inputenc}
\usepackage[english]{babel}
\usepackage{geometry}
  \geometry{
    a4paper,
    total = {170mm, 257mm},
    left = 20mm,
    top = 20mm,
  }
\usepackage{graphicx}
\usepackage{titling}
\usepackage{hyperref}

\setlength{\headheight}{14pt}

\usepackage{amsmath}
\usepackage{amsthm}
\usepackage{amssymb}

% Code environment 
\usepackage{minted}
\usepackage{caption}

\setminted{
  mathescape,
  linenos,
  numbersep=3pt,
  frame=lines,
  framesep=1mm,
  baselinestretch=1,
  fontsize=\footnotesize,
  breaklines=true,
  xleftmargin=1em,
  xrightmargin=0em,
}

\newcounter{code}
\newcommand{\listofcode}{\listof{code}{List of Codes}}
\captionsetup[code]{labelfont=bf, name=Code}

\newcommand{\codefile}[4][python]{%
  \refstepcounter{code}%
  \captionsetup[listing]{name=Code}%
  \captionof{listing}{#2}%
  \label{#3}%
  \inputminted{#1}{#4}%
  }

% Pseudocode environment
\usepackage[ruled,vlined]{algorithm2e}

% Math environments
\usepackage{aliascnt}
\renewcommand{\theequation}{\thesection.\arabic{equation}}

\newenvironment{iproof}[1][Proof idea]
  {
    \begin{proof}[#1]
      \renewcommand{\qedsymbol}{}
  }
  {
  \end{proof}
  }

% Commands 
\usepackage{xparse}

\newcommand{\N}{\mathbb{N}}
\NewDocumentCommand{\R}{g g}{
  \IfNoValueTF{#1}
    {\mathbb{R}}
    {
    \IfNoValueTF{#2}
      {\mathbb{R}^{#1}}
      {\mathbb{R}^{#1 \times #2}}
    }
}

% Document
\usepackage{fancyhdr}
  \fancypagestyle{plain}{
    \fancyhf{}
    \fancyhead[L]{\thetitle}
    \fancyhead[R]{\theauthor}
  }

\makeatletter
\newcommand{\subtitle}[1]{\gdef\@subtitle{#1}}
\newcommand{\@subtitle}{}
\makeatother

\makeatletter
\def\@maketitle{
  \newpage 
  \null 
  \vskip 1em 
  \begin{center}
    \let \footnote \thanks 
      {\LARGE \@title \par}
      \vskip 0.5em
      {\large \@subtitle \par}
      \vskip 1em
      {\large \@date}
  \end{center}
  \par 
  \vskip 1em 
}
\makeatother


\title{CatioCrypto 2025: Multi-Party Computation}
\author{Emanuel Nicolás Herrador}
\newcommand{\speaker}{Eduardo Soria-Vázquez}
\date{26-27 Septiembre 2025}

\begin{document}
  \maketitle 
  \noindent\begin{tabular}{@{}ll}
    Autor & \theauthor \\
    Speaker & \speaker
  \end{tabular}

  \section{Introducción}
  Algunos \underline{ejemplos} acerca de qué se ocupa la criptografía son:
  \begin{itemize}
    \item \textit{Verifiable Computation}: Demostrar que se ha ejecutado un programa,
      sin que quien lo verifique tenga que re-ejecutarlo.
    \item \textit{Zero-Knowledge Proofs}: Similar a VC, pero quien demuestra 
      hacer el cálculo pueden mantener algunos datos ocultos.
    \item \textit{Secret Sharing}: Fragmentar un secreto entre varias entidades,
      de tal forma que solo pueda reconstruirse si ciertos subconjuntos
      de ellas acuerdan hacerlo.
    \item \textit{Multi-Party Computation} (MPC): Calcular sobre datos ``sin verlos'',
      de forma distribuida.
    \item Otros: privacidad diferencial (DP), recuperación privada de información (PIR),
      cifrado totalmente homomórfico (FHE), cifrado funcional (FE), time-lock puzzles,
      blockchaings, criptomonedas, ...
  \end{itemize}

  \section{Multi-Party Computation}
  \subsection{Protocolo MPC}
  Un protocolo MPC consta de $n$ entidades que desconfían las unas de las otras y de modo
  que cada una pueda aportar un valor $P_i$ secreto.
  La idea/objetivo es calcular un función $f$ aplicada a estos valores en conjunto pero sin mostrarlos 
  a otra de las entidades.

  Una forma de resolverlo sin MPC es con una tercera parte de confianza, pero el 
  problema es que no siempre existe.
  Lo que vuelve difícil a MPC es la presencia de un adversario 
  que participa y corrompe a ciertas entidades para coordinar su ataque:
  \begin{itemize}
    \item Corrupción pasiva: las entidades corruptas hacen lo que deben pero comparten 
      información entre ellas para tratar de aprender más cosas sobre las demás entidades 
    \item Corrupción activa: las entidades corruptas pueden actuar de forma arbitraria 
      para sabotear los objetivos de seguridad
  \end{itemize}
  El \underline{objetivo} es lograr todas las propiedades que lograríamos con una tercera 
  parte de confianza pero sin ella.

  \subsection{Primitivas}
  Algunas de las prmitivas usadas por MPC son:
  \begin{itemize}
    \item Garbled Circuits
    \item Secret Sharing 
    \item Oblivious Transfer 
    \item Homomorphic Encryption
  \end{itemize}

  \subsubsection{Secret Sharing (compartición de secretos)}
  \begin{definition}[Esquema de compartición de secretos]
    Sea $\F$ un cuerpo finito y $n,t\in \Z : 0\leq t < n$.
    Sean las entidades $P_1,\dots,P_n$ y un compartidor en posesión de un 
    secreto $s\in\F$.
    Un esquema $(n,t)$-umbral de compartición de secretos consta de las siguientes fases:
    \begin{enumerate}
      \item Fase de compartición: El Compartidor fragmenta $s$ en $n$ partes $s_1,\dots,s_n\in\F$
        y la entidad $P_i$ recibe su parte $s_i\in\F$ 
      \item Fase de reconstrucción: Al menos $t+1$ entidades envían su parte a quien quieran 
        que reconstruya $s$.
    \end{enumerate}

    Y tal que satisface las siguientes propiedades:
    \begin{itemize}
      \item $t$-privacidad: Cualquier conjunto de a lo sumo $t$ partes no revela ninguna 
        información sobre el secreto $s\in\F$
      \item $(t+1)$-reconstrucción: Cualquier conjunto de $t+1$ partes determina de forma 
        única el secreto $s\in\F$
    \end{itemize}
  \end{definition}
  \begin{definition}[Esquema lineal de compartición de secretos]
    Decimos que el esquema es \textit{lineal} si además satisface que $\forall s,t\in\F$
    compartidos como $(s_1,\dots,s_n),(t_1,\dots,t_n)\in\F^n$, tenemos que 
    $(s_1+t_1,\dots,s_n+t_n)$ es una compartición de $s+t\in\F$.
  \end{definition}
  \begin{remark}
    Abusaremos notación y \underline{nos restringiremos} a esquemas de compartición de secretos 
    que sean sobre un cuerpo finito $\F$, umbrales y lineales (sobre $\F$).
    Además, usaremos la notación definida como $[s] := (s; s_1,\dots,s_n)$, de modo que 
    $[s] + [t] = [s+t]$.
  \end{remark}

  \paragraph{Compartición de secretos aditiva}
  Sea $F_p := (\Z/p\Z, +, \cdot),\ \Z/p\Z = (0,\dots,p-1),\ p$ nro. primo, el esquema es:
  \begin{enumerate}
    \item Fase de compartición:
      \begin{enumerate}
        \item Compartidor escoge de forma uniformemente aleatoria $s_1,\dots,s_n\in\F_p : s_1+\dots+s_n=s$.
        \item Compartidor envía la parte $s_i$ a $P_i$
      \end{enumerate}
    \item Fase de reconstrucción:
      \begin{enumerate}
        \item Cada $P_i$ envía $s_i$ a la entidad que quieren que reconstruya el secreto 
        \item Dicha entidad calcula $s=s_1+\dots+s_n$
      \end{enumerate}
  \end{enumerate}
  \begin{remark}
    Notar que el umbral $t$ debe ser $n-1$ para que sea reconstruible teniendo todos los 
    pedacitos de secretos.
    El hecho de que no sea reconstruible con $n-1$ entidades se reduce al OTP porque si 
    tenemos $s_1,\dots,s_{n-1}\in\F_p$ elegidos uniformemente aleatorios, entonces 
    $s_n = s - (s_1+\dots+s_{n-1})$ es también uniformemente aleatorio.
  \end{remark}
\end{document}
